\sectionnn{Introduction}

Dans le cadre de mon stage au sein de l'entreprise padoa, j’ai eu l’opportunité de participer à un projet de déploiement pour un service de santé au travail. Ce projet porte spécifiquement sur la migration des données précédemment stockées dans son ancien logiciel préventiel, vers notre logiciel, récemment adopté par le client SSTI03 pour améliorer le suivi de la santé de ses salariés. 

Cette migration soulève cependant une problématique centrale : comment assurer la continuité et l’exactitude des actions de suivi médical tout en préservant l’intégrité et la traçabilité de données issues de deux systèmes distincts ? Les médecins du travail utilisent ces informations pour orienter leurs décisions et ajuster les stratégies de prévention, ce qui confère une importance particulière à la qualité et à la fiabilité des données intégrées danspadoa. La différence de structures entre les deux logiciels, ainsi que le risque d’incompatibilités, de pertes de données ou d’erreurs de transfert, constituent des défis majeurs. 

Face à cette problématique, notre objectif principal est d’assurer un transfert exhaustif et sécurisé des données, afin que les médecins du travail disposent d’une base complète et conforme aux normes. Par ailleurs, il s’agit de garantir que les données intégrées conservent leur cohérence et leur intégrité, de manière à être aisément accessibles et utilisables dans l’interface de Padoa.
Pour ce rapport, nous nous concentrerons précisément sur la reprise des examens complémentaires en attente de résultat pour illustrer le processus de reprise de données. Ce travail a été effectué plusieurs fois sur des ressources différentes mais le processus reste néanmoins le même. 

Ce rapport de stage présente en détail le déroulement de ce projet d’intégration de données, en exposant les étapes clés : l’analyse des besoins et la préparation des données, le transfert proprement dit, puis la vérification de leur conformité et de leur accessibilité dans le nouvel environnement padoa. Il s’attache également à décrire les défis techniques rencontrés au cours de cette migration, ainsi que les solutions mises en place pour surmonter ces obstacles tout en respectant les exigences de qualité et de sécurité propres aux données de santé.

\smallskip