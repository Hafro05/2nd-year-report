
\sectionnn{Conclusion}

Ce stage a été une expérience particulièrement enrichissante, mêlant défis techniques, travail en équipe et interactions avec le client. En participant à la reprise des données du SSTI03, j’ai pu m’impliquer dans des étapes clés d’un déploiement, depuis l’exploration des bases de données du client jusqu’à la phase de recette. Ces moments, en particulier la recette, ont été l’occasion de mesurer concrètement l’impact de notre travail sur les utilisateurs finaux, comme les médecins, secrétaires et infirmiers.

Intégrer l’équipe intégration m’a permis de découvrir des processus complexes et stratégiques, comme la mise en place des extracteurs, la validation des mappings et l’application de règles spécifiques. J’ai également pu constater l’importance de la collaboration entre les différentes équipes – ops, intégration, et SRE – pour garantir le succès d’un déploiement. Les discussions régulières avec les ops et les points hebdomadaires m’ont apporté une meilleure compréhension des enjeux opérationnels et des attentes du client.

D’un point de vue personnel, ce stage m’a permis de renforcer mes compétences techniques, notamment en manipulation de données, en automatisation des tests avec \textit{pytest}, et en gestion des pipeline. J’ai aussi développé des compétences interpersonnelles essentielles, comme la communication, l’écoute et l’adaptabilité, grâce aux échanges avec mes collègues et les interactions avec le client.

Cette expérience m’a non seulement permis d’approfondir mes connaissances techniques, mais elle m’a aussi donné une vision globale de la gestion d’un projet en entreprise. Je termine ce stage avec un sentiment d’accomplissement et la conviction que les compétences acquises ici seront des atouts précieux dans la suite de mon parcours professionnel.


