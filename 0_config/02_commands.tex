% Commandes pour les étudiants et tuteurs
\newcommand{\addstudent}[3]{\small{#1} & \small{#2} & \small{\href{mailto:#3}{#3}}\\}
\newcommand{\addtutor}[3]{\small{#1} & \small{#2} & \small{\href{mailto:#3}{#3}}\\}

% Informations sur le projet
\newcommand{\firstauthor}{Prénom Nom}
\newcommand{\firstauthorinitials}{P. NOM}
\newcommand{\projecticonl}{path/to/iconl.png}
\newcommand{\projecticonr}{path/to/iconr.png}

% Définir la commande \authorname
\newcommand{\numauthors}{3}
\newcommand{\authorname}{\firstauthor\ifthenelse{\numauthors > 1}{ et al.}{}}

% Tabulation personnalisée
\newcommand\tab[1][0.6cm]{\hspace*{#1}}

% Chapitres et sections sans numérotation
\newcommand{\chapternn}[1]{\chapter*{#1}\addcontentsline{toc}{chapter}{#1}}
\newcommand{\sectionnn}[1]{\section*{#1}\addcontentsline{toc}{section}{#1}}
\newcommand{\subsectionnn}[1]{\subsection*{#1}\addcontentsline{toc}{subsection}{#1}}
\newcommand{\subsubsectionnn}[1]{\subsubsection*{#1}\addcontentsline{toc}{subsubsection}{#1}}

% Types de colonnes personnalisés
\newcolumntype{L}[1]{>{\raggedright\arraybackslash\hspace{0pt}}p{#1}}
\newcolumntype{R}[1]{>{\raggedleft\arraybackslash\hspace{0pt}}p{#1}}
\newcolumntype{C}[1]{>{\centering\arraybackslash\hspace{0pt}}p{#1}}

% Numérotation des sections
\renewcommand\thesection{\arabic{section}}
\renewcommand\thesubsection{\thesection.\arabic{subsection}}

% Configuration des hyperliens
\hypersetup{
    colorlinks      = false,
    linkbordercolor = {1 1 1},
    breaklinks      = true,
    urlcolor        = blue,
    linkcolor       = black,
    citecolor       = black,
    pdftitle        = {Security assessment of connected objects in the Internet of Things : State of the art},
    pdfauthor       = {},
    pdfsubject      = {}
}

% Configuration de l'en-tête et du pied de page
\RequirePackage{fancyhdr}
\pagestyle{fancy}

% Commandes personnalisées pour les couleurs
\newcommand{\validcolor}[1]{\textcolor{green!60!black}{#1}}
\newcommand{\flatcolor}[1]{\textcolor{blue!60!black}{#1}}
\newcommand{\unknowncolor}[1]{\textcolor{orange!80!black}{#1}}
\newcommand{\invalidcolor}[1]{\textcolor{red!60!black}{#1}}